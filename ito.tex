
\section{It\^o's Lemma}

We want to find the present value of derivative payoffs. E.g. an EU call
$g(T, S_T)$. It can be shown that $PV_t=f(t, S_t)$. If we can solve $f$ we are
golden. It\^o's Lemma gives us the stochastic differential equation for the
unknown function, $f$. Theoretically, we can solve the SDE. Put another way,
IL gives us the change in the PV of an option given time and the stock price.
It is the total derivative of the unknown function, $f$, that gives us the PV
of the option.
\[df(t,x)=f_1(t,x)dt+f_2(t,x)dx\]
The total derivative is: $\Delta f$ given the $\Delta t$ with constant $x$ +
$\Delta f$ given the $\Delta x$ with constant $t$.

Given diffusion $Y$;
\begin{equation} \label{IL}
    df(t, Y_t) = \left[ f_1(t,Y_t)+f_2(t,Y_t) \mu_Y (t,Y_t)+ \frac{1}{2} f_{22}
    (t,Y_t) \sigma_Y^2 (t,Y_t) \right] dt+f_2(t,Y_t)\sigma_Y (t,Y_t)dW_t
\end{equation}
where
\[f_1(t, Y_t)=\frac{df(t,Y_t)}{dt}\]
\[f_2(t, Y_t)=\frac{df(t,Y_t)}{dY_t}\]
\[f_{22}(t, Y_t)=\frac{d^2f(t,Y_t)}{dY_t^2}\]

\subsection{Example: Stock Price Process}

We define returns as the log of the price, $R_t=ln(S_t)$. We can use IL to
prove that $S$, from equation \ref{S}, has the solution:
\[
    S_t=S_0e^{\int_0^t\mu_S(u)-\frac{1}{2}\sigma_S^2(u)du+
    \int_0^t\sigma_S(u)dW_u}
\]
Identification
\[f(t,Y_t)=R_t=ln(S_t)\]
\[Y_t=S_t\]
Calculation
\[f_1(t,S_t)=0\]
\[f_2(t, S_t)=\frac{1}{S_t}\]
\[f_{22}(t,S_t)=-\frac{1}{S_t^2}\]
Substitution
\[
    dR_t=\left[ 0+\frac{1}{S_t}\mu_S(t)S_t+\frac{1}{2}\frac{-1}{S_t^2}
    \left[ \sigma_S(t)S_t \right] ^2 \right] dt+ \frac{1}{S_t}\sigma_SS_tdW_t
\]
Simplification
\[
    dR_t= \left[ \mu_S(u)-\frac{1}{2}\sigma_S^2(t) \right] dt + \sigma_S(t)dW_t
\]
$dR_t$ is a generalized Wiener process, equation \ref{dGenW}.

\subsection{Example: Buy and Hold}

We claimed
\[\phi_tS_t + \phi_0S_0=\int_0^t\phi_udG_u\]
where $\phi_t=e^{\int_0^t\delta_S(u)du}$ and $\delta_S(t, S_t)=\delta_S(t)S_t$
because we reinvest dividends.
IL can give us the dynamics of the position.
\[Z_t=\phi_tS_t=e^{\int_0^t\delta_S(u)du}S_t\]
Remembering that
\[\frac{d}{dt}f(x)g(x)=f'(x)g(x)+f(x)g'(x)\]
Identification
\[f(t, Y_t)=e^{\int_0^t\delta_S(u)du}S_t\]
\[Y_t=S_t\]
Calculation
\[f_1=S_te^{\int_0^t\delta_S(u)du}\frac{d}{dt}\int_0^t\delta_S(u)du\]
When differentiating w.r.t. the limits of an integral we can evalutate at the
upper limit if the lower limit does not depend on, in our case, $t$.
\[f_1=S_te^{\int_0^t\delta_S(u)du}\delta_S(t)=\phi_tS_t\delta_S(t)\]
\[f_2=e^{\int_0^t\delta_S(u)du}1=\phi_t\]
\[f_2=0\]
Substitution
\[
    dZ_t=\left[ \phi_tS_t\delta_S(t) + \phi_t\mu_S(t)S_t +\frac{1}{2}0  \right]
    dt + \phi_t\sigma_S^2(t)S_tdW_t
\]
Simplification (where does the squared term go?)
\[
    Z_t=\phi_tS_t=z_0+\int_0^t\phi_tS_t\delta(u)+
    \phi_u\mu_S(u)S_udu+\int_0^t\phi_u\sigma_S(u)S_udW_u
\]
\[
    z_0+\int_0^t\phi_u\mu_S(u)S_udu+
    \int_0^t\phi_u\sigma_S(u)S_udW_u+\int_0^t\phi_tS_t\delta(u)du
\]
\[z_0+\int_0^t\phi_udS_u+\int_0^t\phi_udD_u\]
\[z_0+\int_0^t\phi_ud(S_u+D_u)\]
