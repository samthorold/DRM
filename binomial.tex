
\section{Binomial Intution}

Option pricing is discounting future cash flows to find the present value of an
asset. Instead of modifying the discount rate, think corporate finance, we
modify the expected cash flows using risk adjusted probabilities and discount
at the risk free rate. The economics behind this comes from the
\textbf{replicating portfolio}.

By replicating an asset we can find its price through the law
of one price. Two assets with the same cash flows should have the same
price, otherwise there is arbitrage and there will be much wailing and gnashing
of teeth. If markets are complete, as many independent assets as there
are states of the world, we can always replicate an asset.

Let's say we have some cash flow, X, a stock, S, and a bank account paying R
for sure. We want to combine a postion in the stock and the bank account such
that the payoff from the combined stock and bank account matches the cash flow
in all states of the world.
\[\Delta S+\theta R=X\]
\[uS_0+\theta R=X_u\]
\[dS_0+\theta R=X_d\]
\[P_0=\frac{1}{R}[qX_u+(1-q)X_d]$$ where $$ q=\frac{R-d}{u-d}\]

We can make trees from many of these two state possibilities and account for
more than two possible outcomes using the same simple ideas of a replicating
portfolio and the law of one price. No matter how fine we make our trees we
will always have some imprecision. Taking the number of nodes in the tree to
the extreme is a probability distribution. This leads us to the first building
block in an asset price process.
