
\section{Forward and Spot Measures Black-Scholes}

We can think of the payoff from an EU call option as paying the strike to
receive the stock at time $T$. We will pay the strike if the stock is worth
more.
\[c_T=max(S_T-K,0)=S_T1_{S_T>K}-K1_{S_T>K}\]
We will refer to the first term as $c_{1T}$ and the second as $c_{2T}$.
We can use the stock measure to find the present value of $c_{1T}$.
\[c_{1t}^S=E_t^{Q_S}[c_{1T}^S]\]
iff
\[\frac{c_{1t}}{S_t^c}=E_t^{Q_S}\left[\frac{S_T1_{S_T>K}}{S_T^c}\right]\]
\[
  \frac{c_{1t}}{S_te^{\int_0^t\delta_S(u)du}}=
  E_t^{Q_S}\left[\frac{S_T}{S_Te^{\int_0^T\delta_S(u)du}}1_{S_T>K}\right]
\]
\[
  c_{1t}=S_te^{\int_0^t\delta_S(u)du}\frac{1}{e^{\int_0^T\delta_S(u)du}}
  E_t^{Q_S}\left[ 1_{S_T>K}\right]
\]
\[c_{1t}=S_te^{-\int_t^T\delta_S(u)du}Q_S(S_T>K)\]
The expectation of an indicator function is the probability, under $Q_S$
probabilities here, of that event happening.
Remembering our solution for a geometric Wiener
\[
  S_T=S_te^{\int_t^Tr_u-\delta_S(u)+\sigma_S^2(u)-\frac{1}{2}\sigma_S^2(u)du+
  \int_t^T\sigma_S(u)dW_u^S}=S_te^{m+\sigma\varepsilon}
\]
Where $\varepsilon\sim N(0,1)$.
Note the adjusted Wiener process.
\[Q_S(S_T>K)=Pr(S_te^{m+\sigma\varepsilon}>K)\]
Solve for $\varepsilon$ inside $Pr$
\[
  Q_S(S_T>K)=
  Pr\left(\varepsilon>\frac{ln\left(\frac{K}{S_t}\right) -m}{\sigma}\right)
\]
Remembering section \ref{sec:BS} where we derived the Black-Scholes formula the
first time, we want to find the probability that $\varepsilon$ is less than
some value so we can use the cumulative normal density function. We flip the
inequality
\[
  Pr\left(\varepsilon>\frac{ln\left(\frac{K}{S_t}\right) -m}{\sigma}\right)=
  Pr\left(\varepsilon<-\frac{ln\left(\frac{K}{S_t}\right) -m}{\sigma}\right)
\]
\[=Pr\left(\varepsilon<\frac{ln\left(\frac{S_t}{K}\right) +m}{\sigma}\right)\]
\[
  =Pr\left(\varepsilon<\frac{ln\left(\frac{S_t}{K}\right) +
  \int_t^Tr_u-\delta_S(u)+\frac{1}{2}\sigma_S^2(u)du}
  {\sqrt{\int_t^T\sigma_S^2(u)du}}\right)
\]
Where $\frac{1}{2}\sigma_S^2=\sigma_S^2-\frac{1}{2}\sigma_S^2$. We
recognise this as $d_1$.
\begin{equation} \label{c1t}
  c_{1t}=S_te^{-\int_t^T\delta_S(u)du}N(d_1)
\end{equation}

We can use the forward measure to find the present value of $c_{2T}$.
\[c_{2t}^{P(t,T)}=E_t^{Q_T}[c_{2T}]\]
\[\frac{c_{2t}}{P(t,T)}=KE_t^{Q_T}[1_{S_T}>K]\]
\[c_{2t}=P(t,T)Q_T(S_T>K)\]
In the Black-Scholes world, interest rates are deterministic so $Q_T$ = $Q$.
\[Q_T(S_T>K)=Q(S_T>K)=N(d_2)\]
Remembering $P(t,T)=e^{-\int_t^Tr_udu}$
\begin{equation} \label{c2t}
  c_{2t}=Ke^{-\int_t^Tr_udu}N(d_2)
\end{equation}
