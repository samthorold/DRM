\documentclass[12pt]{article}

\usepackage{amsfonts}
\usepackage{amsmath}
\usepackage[top=1in, bottom=1in, left=1in, right=1in]{geometry}
\usepackage{graphicx}

\graphicspath{ {figures/} }

\begin{document}

\title{NHH - Spring 2017 - ECO423}
\author{Sam Thorold}
\date{\today}
\maketitle

\tableofcontents
\pagebreak

\section{Binomial Intuition}

Option pricing is discounting future cash flows to find the present value of an
asset. Instead of modifying the discount rate, think corporate finance, we
modify the expected cash flows using risk adjusted probabilities and discount
at the risk free rate. The economics behind this comes from the
\textbf{replicating portfolio}.

By replicating an asset we can find its price through the law
of one price. Two assets with the same cash flows should have the same
price, otherwise there is arbitrage and there will be much wailing and gnashing
of teeth. If markets are complete, as many independent assets as there
are states of the world, we can always replicate an asset.

Let's say we have some cash flow, X, a stock, S, and a bank account paying R
for sure. We want to combine a postion in the stock and the bank account such
that the payoff from the combined stock and bank account matches the cash flow
in all states of the world.
$$\Delta S+\theta R=X$$
$$uS_0+\theta R=X_u$$
$$dS_0+\theta R=X_d$$
$$P_0=\frac{1}{R}[qX_u+(1-q)X_d]$$ where $$ q=\frac{R-d}{u-d}$$

We can make trees from many of these two state possibilities and account for
more than two possible outcomes using the same simple ideas of a replicating
portfolio and the law of one price. No matter how fine we make our trees we
will always have some imprecision. Taking the number of nodes in the tree to
the extreme is a probability distribution. This leads us to the first building
block in an asset price process.

\pagebreak

\section{Asset Price Model}

\subsection{Standard Wiener Process, $W$}

The standard Wiener process, W, is the whole stochastic process or law of
motion. Also referred to as Brownian motion after the biologist who first came
up with the informal idea while trying to describe the movements of pollen in
liquid. Wiener formalized the idea.
\begin{itemize}
    \item $W_t$: value at time t, unknown before t, often random variable
    \item $W_{\omega}$: particular path given state $\omega$
    \item $W_{t,\omega}$: value at time t, given state $\omega$
\end{itemize}
$W$ has the following properties;
\begin{enumerate}
\item $W_0=0$: Normalized to 0
\item $W_t-W_s \sim N(0, t-s)$: variance is equal to the time increment
\item Non-overlapping time increments of W are independent
\item W has continuous sample paths, $W_{t, \omega}$ is continuous in t. This
    is possible because, while the normal distribution can jump to an infinite
    number of values, as the increments between jumps become smaller the
    variance converges to 0.
\end{enumerate}

Given the above properties, $W_{t,\omega}$ is not differentiable in t. This is
important because if we could find the derivative we would know which way the
process would move next and trade based on this knowledge. Not consistent with
no arbitrage. This opportunity would be traded away anyway.

We are not finished because W can be negative. This does not fit well with a
stock price process due to limited liability. Someone owning a stock can lose a
maximum of the amount they invested and no more. For a physical asset or
interest rate the possibility of a negative value is not a complete no go, but
we will focus on stocks.

\subsection{Generalized Wiener Process, $X$}
The standard Wiener process is just a series of numbers drawn from a
distribution (white noise from ECO403). We can generalize this somewhat by
adding a deterministic drift component and a deterministic uncertainty scaling
to the source of uncertainty provided by $W$. In differential form this looks
like:
\begin{equation} \label{dGenW}
dX_t=\mu_X(t)dt+\sigma_X(t)dW_t
\end{equation}
where $\mu_X(t)dt$ is the drift, $\sigma_X(t)$ is the uncertainty scaling and
$dW_t$ is the source of uncertainty (our increments of W from above
$\sim N(0, t-s)$).
In descrete time this looks like:
\begin{equation}
X_{t+\Delta t}-X_t=\Delta X_t=\mu_X(t)\Delta t+\sigma_X(t)\Delta W_t
\end{equation}
Since $E[\Delta W_t]=0$, Expected value of $\Delta X_t$ is
\begin{equation} \label{ExpDesGenW}
E[\Delta X_t]=\mu_X(t)\Delta t
\end{equation}
Since variance of a constant is 0 and the variance of the standard Wiener
process, W, is $t-s=\Delta t$ AND the two terms are independent of each other,
we can ignore the covariance (and remembering that constants taken outside of
$Var(...)$ must be squared)
\begin{equation} \label{VarDesGenW}
Var(\Delta X_t)=\sigma^2_X(t)\Delta t
\end{equation}
Linear combinations of normal random variables are normal. Therefore,
\begin{equation}
\Delta X_t \sim N(\mu_X(t)\Delta t, \sigma^2_X(t)\Delta t)
\end{equation}
\begin{equation}
dX_t \sim N(\mu_X(t)dt, \sigma^2_X(t)dt)
\end{equation}

If we wanted to solve X we could devide through by dt. Problem is
$\frac{dW_t}{dt}$ does not exist because otherwise we could trade based on the
derivative and then there would be arbitrage and wailing and gnashing of teeth.
We can write $dX_t$ in integral form
\begin{equation} \label{GenW}
X_t=x+\int_0^t\mu_X(u)du+\int_0^t\sigma_X(u)dW_u
\end{equation}
The funny $dW_u$ means this is an It\^o integral. In fancy language, ``The time
period is transformed by the stochastic process." I do not know what this
means. Basically, the point here is that we cannot use regular integration to
find $X_t$. But, someone clever called It\^o has got our backs here as we'll
see later. Nevertheless, we can say something about the expected value and
variance of $X_t$ using descrete time intuition as we did for $dX_t$.

The expected value of $X_t$ is the expected value of $x$, which we just set
ourselves, plus all the drift terms multiplied by the time interval. We can
ignore the dispersion term because $W_t$ comes from a distribution with mean 0.
We are summing up the expectation of $\Delta X_t$ we came up with in equation
\ref{ExpDesGenW}.
\begin{equation}
E[X_t]=E[x]+E \left[ \sum \mu_X(u)\Delta u\right] +0
\end{equation}

The deterministic terms have no variance so we can ignore them. The increments
of $W$ are independent so we do not need to think about covariances here. The
variance of a sum is the sum of the variances. We can sum up the descrete time
variances of $\Delta X_t$ we came up with above (remembering the $\sigma_X^2$
was introduced when the deterministic uncertainty scaling component
$\sigma_X(t)$ was taken outside of the $Var(...)$ term and so had to be
squared in equation \ref{VarDesGenW}. $\Delta t=t-s$, $\Delta u$ takes the
place of $\Delta t$ here.)
\begin{equation}
Var(X_t)=\sum \sigma^2(u)\Delta u
\end{equation}

Linear combinations of normal random numbers are normal. Putting this back
into continuous time gives us:
\begin{equation}
X_t \sim N \left( x+ \int_0^t \mu_X(u)du, \int_0^t \sigma_X^2(u)du \right)
\end{equation}

X has inherited; independent increments and continuous, non-differentiable
sample paths. We have not finished. X can still be negative which is not
consistent with limited liability of stocks. Furthermore, we still cannot solve
the funny It\^o integral $\frac{dW_u}{du}$. We need to use a process for which
a solution to the It\^o integral has already been found that has the properties
that we have just described above; drift, uncertainty scaling, a source of
uncertainty and independent increments.

\subsection{Diffusion, $Y$}
Diffusions have been well studied by physicists so we can borrow their hard
work in solving the equations we also borrow from them. In differential form:
$$dY_t=\mu_Y(t, Y_t)dt+\sigma_Y(t, Y_t)dW_t$$
In integral form:
$$Y_t=y+\int_0^t\mu_Y(u, Y_u)du+\int_0^t\sigma_Y(u,Y_u)dW_u$$
$dY_t$ is locally normal because linear combination of normals. The whole
process is not normal. $\mu$ (drift) and $\sigma$ (uncertainty scaling -
dispursion) are known functions of $t$ and $Y_t$.

It can be shown, not by me, that $Y$ is Markovian. That is to say, it has no
memory (only depends on $t$ and $Y_t$). The economic interpretation of this is
that all information about the asset is reflected in the price - market
efficiency. This is an important theoretical property even if we dispute the
efficient market hypothesis.

BUT we still have not solved our $\frac{dW_t}{dt}$ problem. Mr. It\^o will save
us.

\subsection{Stock Price Model, $S$}
We will use a particular case of a diffusion as our stock price model, $S$. This
allows us to combine the properties of the generalized wiener process, $X$, and
a diffusion, $Y$.
\begin{itemize}
\item deterministic drift, $\mu$
\item deterministic uncertainty scaling, $\sigma$
\item random (stochastic?) source of uncertainty, $dW_t$
\item particular path non-differentiable in t
\item independent increments
\item no memory
\end{itemize}

$$dS_t=\mu_S(t)S_tdt+\sigma_S(t)S_tdW_t$$
Here we have already said something about the drift and volatility scaling
terms. In a vanilla diffusion, $Y$, the drift, $\mu_Y(t, Y_t)$, depends on both
the time and level of the diffusion. Our stock model has assumed the drift only
depends on the time and is multiplied by the stock level.
$$\frac{dS_t}{S_t}=\mu_S(t)dt+\sigma_S(t)dW_t$$
$\frac{\Delta S_t}{S_t}=R_t$ so now we have an expression for the period by
period gross return of our stock. $\mu_S(t)dt$ represents the systemic risk. As
usual we want to know the expected value and variance of $\frac{dS_t}{S_t}$.
Since $dW_t \sim N(0,t-s)$:
$$E \left[ \frac{dS_t}{S_t} \right] = \mu_S(t)dt$$
and
$$Var \left( \frac{dS_t}{S_t} \right) = \sigma_S^2(t)Var(dW_t)$$
$$Var \left( \frac{dS_t}{S_t} \right) = \sigma_S^2(t)dt$$
This model addresses the limited liability problem from earlier. $S_t>0$ for
all $t$. We will show later:
\begin{equation} \label{S}
S_t=S_0e^{\int_0^t\mu_S(u)-\frac{1}{2}\sigma_S^2(u)du+
\int_0^t\sigma_S(u)dW_u}
\end{equation}
or, with constant coefficients
\begin{equation} \label{SConsCoef}
S_t=S_0e^{(\mu_S-\frac{1}{2}\sigma_S^2)t+\sigma_SW_t}
\end{equation}
$e^{whatever} > 0$ so we are golden on limited liability. BUT strictly
speaking this does not allow for default, $S_t=0$.

Furthermore, $R_t=ln(S_t)$ means returns are normal since
$$R_t=ln(S_t)+\int_0^t \left[ \mu_S(u)-\sigma_S^2(u) \right] du +
\int_0^t \sigma_S(u)dW_u$$
Recognizing everything after $ln(S_t)$ is a general Wiener process (equation
\ref{GenW}) we remember that linear combinations of normal numbers are normal.
This is important because
\textbf{if $R$ is normal than future returns are independent of past returns
  as well as past AND current current stock price(s)}.

\pagebreak

\section{Gains from Trade, $G$}

\subsection{Dividends}

$\phi$ is the number of stocks. So $\phi_t S_t$ is the value of the stock
position at date t. The value of the position from time $0$ to $t$ =
$\int_0^t \phi_u dS_u$. We have a model for $dS_u$.

Unfortunately, we are forgetting about one common aspect of stocks, dividends.
Since we work in continuous time we will have to model dividends as a
constant drip of income, at a deterministic rate $\delta_S(t)$,  rather than
discrete lumps of cash. While this is not all that realistic for individual
stocks, the model does make sense for a security such as an index fund which
would pay out in something of a drip. There will be a stock in the index
paying some small amount of dividend compared to the value of the index. This
means $S_t$ is an ex-dividend price, we have not taken into account the gains
(or foregone gains) from dividends.
$$G_t=S_t+D_t$$
where
$$D_t=\int_0^t \delta_S(u)S_udu$$
and
$$dG_t=dS_t+dD_t$$
which is a mix of present and accounting values. Over small time interval
$dt$ we don't mind.
$$dG_t=\mu_S(t)S_tdt+\sigma_S(t)S_tdW_t+\delta_S(t)S_tdt$$
$$dG_t=\left[ \mu_S(t)+\delta_S(t) \right] S_tdt+\sigma_S(t)dW_t$$
$G_t$ does not assume optimal use of dividends, i.e. reinvestment. Since we
work with $dG_u$ this is fine.

\subsection{Buy/Hold Strategy}

Assume we invest in a bank account and a stock. Buy and hold so the quantities
of the bank account and the stock are fixed from time $0$ to $t$. We invest
$\theta$ quantity of \$1 bank accounts and $\phi$ quantity of $S$. The bank
account process is
$$A_t=e^{\int_0^t r(u)du}$$
$$dA_t=e^{\int_0^t r(u)du} \int_0^t r(u)du$$
$$dA_t=A_tr_tdt$$
So the portfolio value from $0$ to $t$ is
$$\int_0^t \phi_u dG_u + \int_0^t \theta_udA_u$$

\subsection{Self-financing Portfolio}

Remembering the binomial intuition, we need a self-financing portfolio to
replicate the option payoff.
\begin{equation} \label{replport}
\phi_tS_t+\theta_tA_t=\phi_0S_0+\theta_0A_0+ \int_0^t \phi_u dG_u +
\int_0^t \theta_udA_u
\end{equation}
We don't know how to solve integrals that don't have regular $dt$ or $du$
terms.

\pagebreak

\section{It\^o's Lemma, IL}

\subsection{Intuition}

We want to find the present value of derivative payoffs. E.g. an EU call
$g(T, S_T)$. It can be shown that $PV_t=f(t, S_t)$. If we can solve $f$ we are
golden. It\^o's Lemma gives us the stochastic differential equation for the
unknown function, $f$. Theoretically, we can solve the SDE. Put another way,
IL gives us the change in the PV of an option given time and the stock price.
It is the total derivative of the unknown function, $f$, that gives us the PV
of the option.
$$df(t,x)=f_1(t,x)dt+f_2(t,x)dx$$
The total derivative is: $\Delta f$ given the $\Delta t$ with constant $x$ +
$\Delta f$ given the $\Delta x$ with constant $t$.

\subsection{Result}

Given diffusion $Y$;
\begin{equation} \label{IL}
df(t, Y_t) = \left[ f_1(t,Y_t)+f_2(t,Y_t) \mu_Y (t,Y_t)+ \frac{1}{2} f_{22}
(t,Y_t) \sigma_Y^2 (t,Y_t) \right] dt+f_2(t,Y_t)\sigma_Y (t,Y_t)dW_t
\end{equation}
where
$$f_1(t, Y_t)=\frac{df(t,Y_t)}{dt}$$
$$f_2(t, Y_t)=\frac{df(t,Y_t)}{dY_t}$$
$$f_{22}(t, Y_t)=\frac{d^2f(t,Y_t)}{dY_t^2}$$

\subsection{Example: Stock Price Process}
We define returns as the log of the price, $R_t=ln(S_t)$. We can use IL to
prove that $S$, from equation \ref{S}, has the solution:
$$S_t=S_0e^{\int_0^t\mu_S(u)-\frac{1}{2}\sigma_S^2(u)du+
\int_0^t\sigma_S(u)dW_u}$$
Identification
$$f(t,Y_t)=R_t=ln(S_t)$$
$$Y_t=S_t$$
Calculation
$$f_1(t,S_t)=0$$
$$f_2(t, S_t)=\frac{1}{S_t}$$
$$f_{22}(t,S_t)=-\frac{1}{S_t^2}$$
Substitution
$$dR_t=\left[ 0+\frac{1}{S_t}\mu_S(t)S_t+\frac{1}{2}\frac{-1}{S_t^2}
\left[ \sigma_S(t)S_t \right] ^2 \right] dt+ \frac{1}{S_t}\sigma_SS_tdW_t$$
Simplification
$$dR_t= \left[ \mu_S(u)-\frac{1}{2}\sigma_S^2(t) \right] dt +
\sigma_S(t)dW_t$$
$dR_t$ is a generalized Wiener process, equation \ref{dGenW}.

\subsection{Example: Buy and Hold}

We claimed
$$\phi_tS_t + \phi_0S_0=\int_0^t\phi_udG_u$$
where $\phi_t=e^{\int_0^t\delta_S(u)du}$ and $\delta_S(t, S_t)=\delta_S(t)S_t$
because we reinvest dividends.
IL can give us the dynamics of the position.
$$Z_t=\phi_tS_t=e^{\int_0^t\delta_S(u)du}S_t$$
Remembering that
$$\frac{d}{dt}f(x)g(x)=f'(x)g(x)+f(x)g'(x)$$
Identification
$$f(t, Y_t)=e^{\int_0^t\delta_S(u)du}S_t$$
$$Y_t=S_t$$
Calculation
$$f_1=S_te^{\int_0^t\delta_S(u)du}\frac{d}{dt}\int_0^t\delta_S(u)du$$
When differentiating w.r.t. the limits of an integral we can evalutate at the
upper limit if the lower limit does not depend on, in our case, $t$.
$$f_1=S_te^{\int_0^t\delta_S(u)du}\delta_S(t)=\phi_tS_t\delta_S(t)$$
$$f_2=e^{\int_0^t\delta_S(u)du}1=\phi_t$$
$$f_2=0$$
Substitution
$$dZ_t=\left[ \phi_tS_t\delta_S(t) + \phi_t\mu_S(t)S_t +\frac{1}{2}0  \right]
dt + \phi_t\sigma_S^2(t)S_tdW_t$$
Simplification (where does the squared term go?)
$$Z_t=\phi_tS_t=z_0+\int_0^t\phi_tS_t\delta(u)+
\phi_u\mu_S(u)S_udu+\int_0^t\phi_u\sigma_S(u)S_udW_u$$
$$z_0+\int_0^t\phi_u\mu_S(u)S_udu+
\int_0^t\phi_u\sigma_S(u)S_udW_u+\int_0^t\phi_tS_t\delta(u)du$$
$$z_0+\int_0^t\phi_udS_u+\int_0^t\phi_udD_u$$
$$z_0+\int_0^t\phi_ud(S_u+D_u)$$

\section{Black and Scholes PDE}

If $c_t=PV_t(g(t, S_t))=f(t, S_t)$ then IL gives us the SDE/diffusion for
$c_t$. SDEs are hard to solve. We will use IL to find an SDE that $c_t$ must
satisfy. Using the replicating portfolio intuition, the replicateing portfolio
must match the dynamics of $c_t$ from IL.
\begin{equation}
  dc_t=\left[ c_1+c_2\mu_S(t)S_t+\frac{1}{2}c_{22} \left[ \sigma_S(t)S_t
  \right] ^2 \right] dt + c_2\sigma_S(t)S_tdW_t
\end{equation}
The self-financing replicating portfolio is given by equation \ref{replport}.
The dynamics of the present value of the option must equal the dynamics of the
replicating portfolio.
\begin{equation}
  dc_t=d(\phi_t S_t+\theta_t A_t)=\phi_t dG_t + \theta_t dA_t
\end{equation}
There is no need for IL here we just write the formula in differential form.
\begin{equation}
  dc_t=\phi_t \left[ \mu_S(t)S_tdt + \sigma_S(t)S_tdW_t + \delta_S(t)S_tdt
  \right] + \theta_tr_tA_tdt
\end{equation}

\begin{equation}
  dc_t= \left[ \left[ \mu_S(t) + \delta_S(t) \right] \phi_t S_t + \theta_t r_t
  A_t \right] dt + \phi_t \sigma_S(t)S_tdW_t
\end{equation}
Two diffusions $X$ and $Y$ are equal for all $t$ iff $\mu_X(t, X_t)=\mu_Y(t,
Y_t)$ and $\sigma_X(t, X_t)=\sigma_Y(t, Y_t)$. Collect drift and dispersion
terms for the option and the replicating portfolio:
\begin{equation}
  c_1+c_2\mu_S(t)S_t+\frac{1}{2}c_{22} \left[ \sigma_S(t)S_t \right] ^2 =
  \left[ \mu_S(t) + \delta_S(t) \right] \phi_t S_t + \theta_t r_t A_t
\end{equation}
\begin{equation}
  c_2\sigma_S(t)S_t=\phi_t\sigma_S(t)S_t
\end{equation}
Now we have two equations and two unknowns, $\theta_t$ and $\phi_t$.
\begin{equation}
\phi_t=c_2
\end{equation}
Which means
$$c_1+c_2\mu_S(t)S_t+\frac{1}{2}c_{22} \left[ \sigma_S(t)S_t \right] ^2 =
c_2\mu_S(t)S_t + c_2\delta_S(t)S_t + \theta_t r_t A_t$$
Interestingly, this means the required rate of return, $\mu_S$,
disappears. This does \textbf{not} mean the option price is independent from
the required rate of return. \textbf{$\mu_S$ is implicit in the stock price.}
\begin{equation}
  \theta_t=\frac{1}{r_t A_t} \left[ c_1 - c_2\delta_S(t)S_t + \frac{1}{2}
  c_{22} \left[ \sigma_S(t)S_t \right] ^2 \right]
\end{equation}
Now we can throw $\phi_t$ and $\theta_t$ back into the replicating portfolio.
$$c_t=\phi_tS_t+\theta_tA_t$$
\begin{equation}
  c_t=c_2S_t+\frac{1}{r_tA_t}A_t \left[ c_1-c_2\delta_S(t)S_t + \frac{1}{2}
  c_{22} \left[ \sigma_S(t)S_t \right] ^ 2 \right]
\end{equation}
Simplifying and dropping the $t$ subscripts
$$rc=rc_2S+c_1-c_2\delta_SS+\frac{1}{2}c_{22}\sigma_S^2S^2$$
Rearranging gives us the fundamental PDE (no required rate of return!)
\begin{equation} \label{fundPDE}
  rc=c_1+c_2 \left( r-\delta_S \right) +\frac{1}{2}c_{22}\sigma_S^2S^2
\end{equation}
The fundamental PDE is contract independent. We have not specified any
boundary conditions such as $c(T, S_T)=g(T, S_T)=max(S_T-K, 0)$. All
derivatives satisfy equation \ref{fundPDE}. Whatever drift we throw in we will
end up with $c_2(r-\delta_S)$. It is important to note that while
the solution to $c(t, S_t)$ does not depend on the required rate of return,
the value of the option is not independent of $\mu_S$. The required rate of
return is implicit in the stock price. BUT, how do we solve the fundamental
PDE?

\subsection{Feynman-Kac}

Feynman-Kac helps us solve the fundamental PDE. Equation \ref{fundPDE} is an
instance of a more general PDE:
$$c_1(t,x)+c_2(t,x)a(t,x)+\frac{1}{2}c_{22}(t,x)\sigma^2(t,x)=c(t,x)b(t,x)$$
with the boundary condition: $C(T,x)=g(x)$.
The economics of the replicating portfolio can tell us the structure of these
more general formulas.
FK says the solution to $c(t,x)$ is:
\begin{equation} \label{feynkacGen}
  c(t,x)=E\left[ e^{-\int_t^Tb(u,Z_u)du}g(Z_T) \right]
\end{equation}
and the dynamics is:
$$dZ_u=a(u,Z_u)du+\sigma(u,Z_u)dW_u$$

As an example we can consider an EU call in the Black-Scholes world ($S$). We
know the particular PDE, equation \ref{fundPDE}. Identification:
\begin{itemize}
\item $t\sim t$
\item $x\sim S$
\item $a(t,S)\sim (r-\delta_S(t))S$
\item $b(t,S)\sim r_t$
\item $\sigma(t,S)\sim \sigma_S(t)S$
\item $g(S)\sim max(S_T-K,0)$
\end{itemize}
Substitution:
\begin{equation} \label{feynkacPar}
  c(t,S)=E\left[ e^{-\int_t^Tr(u)du}max(S_T-K,0) \right]
\end{equation}
The value of the option is the expected value of the discounted value at
expiration where the dynamics, $Z=S$, is given by:
\begin{equation} \label{feynkacParDyn}
dZu=(r(u)-\delta_S(u))Z_udu+\sigma_S(u)Z_udW_u
\end{equation}

\subsection{Black-Scholes Derivation} \label{sec:BS}

Recognizing the dynamics (equation \ref{feynkacParDyn}) is a geometric Wiener
process we can write the solution to the PDE as
\begin{equation} \label{BSPDE}
  Z_T=Z_te^{\int_t^T\left[ r(u)-\delta(u) \right]
  -\frac{1}{2}\sigma_S^2(u)du
  + \int_t^T\sigma_S(u)dW_u}
\end{equation}
This is messy but we can simplify notation.
$$\mu=\int_t^Tr(u)-\delta_S(u)-\frac{1}{2}\sigma_S^2(u)du$$
$$\sigma=\sqrt{\int_t^T\sigma_S^2(u)du}$$
$$Z_T=Z_te^{\mu+\sigma\varepsilon}$$
where
$$\varepsilon \sim N(0,1)$$
This is a simpler way of writing $Z_T$ but means the same thing. It is the risk
adjusted stock price process discounted at the risk-free rate.
We are still left trying to solve \ref{feynkacPar}. Remembering that our
risk-free rate is deterministic we can take it outside the expectation.
$$e^{-\int_t^Tr(u)du}E[max(Z_T-K,0)]$$
The remaining $max(Z_T-K,0)$ has a kink. Writing with indicator
functions solves this problem. We can take the expected value of the indicator
functions separately.
$$E\left[ (Z_T-K)1_{Z_T\geq K} \right] $$
$$E\left[ Z_T1_{Z_T\geq K} - K1_{Z_T\geq K} \right] $$
\begin{equation} \label{BSind}
  E\left[ Z_T1_{Z_T\geq K} \right] - KE\left[ 1_{Z_T\geq K} \right]
\end{equation}
The expected value of an indicator function is the probability the event
will happen. Tackling the second part of \ref{BSind} first:
$$E\left[ 1_{Z_T\geq K}\right] =Pr(Z_T\geq K) =
Pr(Z_te^{\mu +\sigma\varepsilon}\geq K)$$
Working with normals is much nicer than working with lognormals. We can solve
for $\varepsilon$ in the inequality. It is also important to note
that \textbf{by substituting $Z_T$ for $Z_te^{\mu+\sigma\varepsilon}$ we
observe $Z_t$ at time $t$.}
This means is it a constant rather than a stochastic variable.
$$Z_te^{\mu +\sigma\varepsilon}\geq K$$
$$e^{\mu +\sigma\varepsilon}\geq\frac{K}{Z_t}$$
$$\mu +\sigma\varepsilon\geq ln\left( \frac{K}{Z_t} \right)$$
$$\varepsilon\geq\frac{ln\left( \frac{K}{Z_t} \right)
-\mu}{\sigma}$$
To use the cumulative normal density function we want to get the probability
that $\varepsilon$ is below a value not above.
This follows from:
$$E\left[ 1_{\varepsilon\leq K} \right] =
\int_{-\infty}^{\infty} 1_{\varepsilon\leq K} n(\varepsilon)d\varepsilon$$
where $n(\varepsilon)$ is the normal probability density function.
We end up here because if $X$ is continuous,
it is not but we make it so with an indicator function,
then the expectation of $g(X)$ is defined as
$$E[g(X)]=\int_{-\infty}^{\infty}g(x)f(x)dx$$
where $f$ is the probability density function of $X$.
We can now flip the inequality and put a minus sign on the RHS.
$$Pr\left( \varepsilon\leq -\frac{ln\left(\frac{K}{Z_t}\right)
-\mu}{\sigma}\right) $$
Minus sign means we flip the $K$ and $Z_T$ because
$$-ln\left(\frac{a}{b}\right) =-(ln(a)-ln(b))=ln(b)-ln(a)=
ln\left(\frac{a}{b}\right)$$
$$Pr\left( \varepsilon\leq\frac{ln\left(\frac{Z_t}{K}\right)
+\mu}{\sigma}\right) $$
$$E\left[ 1_{\varepsilon\leq K} \right] =
N\left( \frac{ln\left(\frac{Z_t}{K}\right)
+\mu}{\sigma}\right) =N(-L)$$
Now tackling the second part of \ref{BSind} in a similar vein:
$$E\left[ Z_T1_{Z_T\geq K} \right] =
\int_{-\infty}^{\infty}Z_te^{\mu+\sigma\varepsilon}1_{Z_T\geq
K}n(\varepsilon)
d\varepsilon$$
We have found
$\int_{-\infty}^{\infty} 1_{\varepsilon\leq K}
n(\varepsilon)d\varepsilon=
\frac{ln\left(\frac{K}{Z_t}\right)-\mu}{\sigma}$. We can call this expression
$L$ and set it as the lower limit of the integral \textbf{why?}, similar to
setting $K$ as the upper limit when finding the expectation of the raw
indicator function - this time we set $L$ as the lower limit because the
inequality sign of the indicator functions are opposite.
$$\int_{L}^{\infty}Z_te^{\mu+\sigma\varepsilon}n(\varepsilon)d\varepsilon$$
Remembering that $Z_t$ is observed at time $t$ and that $n(e)$ is the normal
density function.
$n(e)=\frac{1}{\sqrt{2\pi}}e^{-\frac{\varepsilon^2}{2}}$.
$$Z_t\int_{L}^{\infty}e^{\mu+\sigma\varepsilon}
\frac{1}{\sqrt{2\pi}}e^{-\frac{\varepsilon^2}{2}}d\varepsilon$$
We can combine $\varepsilon$ by completing the square
$(a+b)^2=a^2+2ab+b^2$.
$$\frac{1}{2}(2\mu+2\sigma\varepsilon-\frac{1}{2}\varepsilon^2)=
\frac{1}{2}(2\mu-(\sigma-\varepsilon)^2+\sigma^2)$$
Therefore
$$Z_t\int_{L}^{\infty}e^{\mu+\frac{1}{2}\sigma^2}
\frac{1}{\sqrt{2\pi}}e^{-\frac{1}{2}(\varepsilon-\sigma)^2}d\varepsilon$$
We still need the standard normal density in order to use our $N(...)$. We can
set $u=\varepsilon-\sigma$ therefore, $du=d\varepsilon$ because $\sigma$ is
constant.
When $\varepsilon=L$; $u=L-\sigma$.
When $\varepsilon=\infty$; $u=\infty-\sigma$. Now
$$Z_te^{\mu+\frac{1}{2}\sigma^2}
\int_{L-\sigma}^{\infty}\frac{1}{\sqrt{2\pi}}e^{-\frac{1}{2}u^2}du$$
$$Z_te^{\mu+\frac{1}{2}\sigma^2}
\int_{L-\sigma}^{\infty}n(u)du$$
Now we have our standard normal $\int_{L-\sigma}^{\infty}n(u)du$ and we can
switch the inequality direction as we did above with the same logic.
$$Pr(>L-\sigma)=Pr(<-(L-\sigma))$$
$$Z_te^{\mu+\frac{1}{2}\sigma^2}
\int_{\infty}^{-(L-\sigma)}n(u)du$$
$$E\left[ Z_T1_{Z_T\geq K} \right] =
Z_te^{\mu+\frac{1}{2}\sigma^2}N(-(L-\sigma))$$
Putting this all back together
\begin{equation}
c(t,S)=e^{-\int_t^Tr(u)du}
\left[ Z_te^{\mu+\frac{1}{2}\sigma^2}N(-(L-\sigma)) -KN(-L) \right]
\end{equation}
where; $Z_t=S$,
$$L=\frac{ln\left(\frac{K}{Z_t}\right)-\mu}{\sigma}$$
$$\mu=\int_t^Tr(u)-\delta_S(u)-\frac{1}{2}\sigma_S^2(u)du$$
$$\sigma=\sqrt{\int_t^T\sigma_S^2(u)du}$$
We have been modeling a $Z$ process and not $S$. Option pricing is computing
present values but by adjusting the expected payoff rather than the discount
rate. While $S$ grows at $\mu$, $Z$ grows at $r$.
We can substitute to get to the more familiar Black-Scholes-Merton model.
Remembering our simplifying notation for $\mu$ and $\sigma$
$$\mu=\int_t^Tr(u)-\delta_S(u)-\frac{1}{2}\sigma_S^2(u)du$$
$$\sigma=\sqrt{\int_t^T\sigma_S^2(u)du}$$
$$c(t,S)=
e^{-\int_t^Tr(u)du+\int_t^Tr(u)du-\delta_S(u)-\frac{1}{2}\sigma_S^2(u)du
  +\int_t^T\frac{1}{2}\sigma_S^2(u)du}
Z_tN(-(L-\sigma))-e^{\int_t^Tr(u)du}KN(-L)$$
\begin{equation} \label{BS}
e^{-\delta_S(u)du}Z_tN(-(L-\sigma))-e^{\int_t^Tr(u)du}KN(-L)
\end{equation}
Only $d_1$ and $d_2$ remain unfamiliar now.
$$-(L-\sigma)=-L+\sigma=-\frac{ln\left(\frac{K}{Z_t}\right)-\mu}{\sigma}+
\sqrt{\int_t^T\sigma_S^2(u)du}$$
$$\frac{ln\left(\frac{Z_t}{K}\right)+
\int_t^Tr(u)-\delta_S(u)-\frac{1}{2}\sigma_S^2(u)du}
{\sqrt{\int_t^T\sigma_S^2(u)du}}+
\frac{\int_t^T\sigma_S^2(u)du}{\sqrt{\int_t^T\sigma_S^2(u)du}}$$
\begin{equation} \label{BSd1}
d_1=\frac{ln\left(\frac{Z_t}{K}\right)+
    \int_t^Tr(u)-\delta_S(u)+\frac{1}{2}\sigma_S^2(u)du}
{\sqrt{\int_t^T\sigma_S^2(u)du}}
\end{equation}
$$-L=-\frac{ln\left(\frac{K}{Z_t}\right)-\mu}{\sigma}$$
\begin{equation} \label{BSd2}
d_2=\frac{ln\left(\frac{Z_t}{K}\right)+
\int_t^Tr(u)-\delta_S(u)-\frac{1}{2}\sigma_S^2(u)du}
{\sqrt{\int_t^T\sigma_S^2(u)du}}
\end{equation}
To get the BSM formula with constant parameters we need to remember the rules
for integrating a constant. $\int_a^bkdx=k\int_a^b1dx=k(b-a)$.
$$c(t,S)=e^{-\delta_S(T-t)}SN(d_1)-e^{-r(T-t)}KN(d_2)$$
$$d_1=\frac{ln\left( \frac{S}{K} \right) +
(r-d_S+\frac{1}{2}\sigma_S^2)(T-t)}{\sigma_S\sqrt{T}}$$
$$d_2=\frac{ln\left( \frac{S}{K} \right) +
    (r-d_S-\frac{1}{2}\sigma_S^2)(T-t)}{\sigma_S\sqrt{T}}$$
The most important point here is the \textbf{replicating portfolio}.

\pagebreak

\section{Simulation}

Remembering that Feynman-Kac is taking the expectation of a function subject to
some dynamics, $E[g(Z_T)]$, we can see how this can be applied to simulation.
If we know the distribution of $Z$ then we can use a computer to draw many
$Z$s (iid) and we can take the average $g(Z_T^i)$.
The  central limit theorem says that given enough draws our average value will
be  very close to the true expected value.
$$\frac{1}{k}\sum_{i=1}^kg(Z_T^i)$$
This is the payoff and not the present value. We must still discount.
What is a ``large" number of draws is  not an exact science.
It is important to remember to average the payoffs  not the values of $Z$.
We are not looking for the expected stock price.
A  confidence interval is essential because we are using random numbers.

It is unlikely we know the distribution to draw our $Z$s from. In this case we
can discretize the continuous time function using the Euler scheme.
$$dZ_t\approx\Delta Z_t=Z_{t+\Delta t}-Z_t$$
$$\Delta t\approx (t+\Delta)-t=\Delta t$$
$$dW_t\approx\Delta W_t=W_{t+\Delta t}-W_t=
\sqrt{\Delta t}\varepsilon_{t+\Delta t}$$
where $\varepsilon_{t+\Delta t}\sim N(0,1)$.
The first order Euler scheme is given by
$$Z_{t+\Delta t}=Z_t+\mu_Z(t,Z_t)\Delta t+\sigma_Z(t,Z_t)\sqrt{\Delta t}
\varepsilon_{t+\Delta t}$$
where the $\sqrt{\Delta t}$ comes from the variance of the standard Brownian
motion being proportional to the time interval between steps.
Steps when the marginal distribution is not known;
\begin{enumerate}
  \item Draw a matrix of shocks, [$\varepsilon_1^k$, $\varepsilon_2^k$,
    ..., $\varepsilon_M^k$]
  \item Make $k$ paths with each list of shocks
    $$Z_{m+1}^k=Z_m^k+\mu_Z(m,Z_m^k)\Delta t+\sigma_Z(m,Z_M^k)\sqrt{\Delta t}
    \varepsilon_{m+1}^k$$
    for each $\varepsilon^k$ for each $k$
  \item $g_M^k=g(Z_M^k)$ and
    $c(t,S)=e^{-\int_t^tr(u)du}\frac{1}{k}\sum_{i=1}^kg_M^i$
\end{enumerate}
For any diffusion, $Y$, we can define an Euler scheme
$$Y_{m+1}^k=Y_m^k+\mu_Y(m,Y_m^k)+
\sigma_Y(m,Y_m^k)\sqrt{\Delta t}\varepsilon_{m+1}$$
where $\varepsilon_{m+1}\sim N(0,1)$.

Unfortunately, the discrete time model may not have the same properties as the
continuous time model. One attractive feature of the continuous time stock
price model is the guarantee that the stock price will always be positive as
long as the initial $S$ is positive. For the discrete time model we cannot make
the same guarantee because there is a chance the random shock $\varepsilon$ is
very large at a time that the stock price is low. This undesirable property
appears because the discrete time model does not shrink the variance to $dt$
as in the continuous time model. This non-zero variance means there is a chance
the shock is very large.

One way to avoid this problem, in the specific setting of a stock price model,
is to model the continuous time return process
$$dR_t=(\mu_S-\frac{1}{2}\sigma_S^2)dt+\sigma_SdW_t$$
in discrete time
$$R_{m+1}=(\mu_S-\frac{1}{2}\sigma_S^2)\Delta t
+\sigma_S\sqrt{\Delta t}\varepsilon_{m+1}$$
Since the stock price is given by
$$S_m^k=e^{R_m^k}$$
the stock price level cannot be negative.

Drawing random numbers means any simulation we run is just one realization of
many possible outcomes. It is more informative to create a confidence interval
than come up with one single estimate of a derivative's value. One approach to
smaller standard errors is to use antithetic variates. If we draw a random
number, $u_m$, from a uniform distribution, antithetic variates means we also
use $\tilde{u}_m=1-u_m$. We get the standard normal from the inverse cumulative
distribution function: $\tilde{\varepsilon}_m=F^{-1}(\tilde{u}_m)$. The
standard normal distribution has symmetry so
$$\tilde{\varepsilon}_m=F^{-1}(\tilde{u}_m)=-\varepsilon_m$$
$\tilde{\varepsilon}_m$ and $\varepsilon_m$ are highly correlated. This does
not matter as long as we keep the two series of shocks separate. We can still
use the same techniques on the separate series of shocks and take the averge of
the two estimates for the same option value as our final estimate. Antithetic
variates result in smaller confidence intervals from the same number of random
draws.
$$g_M^k=g(T,Y_M^k(\varepsilon))$$
$$\tilde{g}_M^k=g(T,Y_M^k(\tilde{\varepsilon}))$$
$$\bar{g}_M^k=\frac{1}{2}(g_M^k+\tilde{g}_M^k)$$
$$c(t,y)=e^{-\int_t^tr_udu}\frac{1}{K}\sum_{k=1}^K\bar{g}_M^k$$

\pagebreak

\section{Equivalent Martingale Measures}

So far we have been working implicitly with a physical probability measure,
$P$. From now on we will risk-adjust $P$ to get a risk-adjusted probability
measure, $Q$. $Q$ is an equivalent martingale measure relative to $P$ if;
\begin{enumerate}
  \item $P[x]=0$ when $Q[x]=0$. That is to say, $P$ and $Q$ agree what cannot
    happen. ``Equivalence"
  \item \begin{equation} \label{EMM}
      G_t^{n*}=E_t^Q[G_T^{n*}]
    \end{equation}
    That is to say, all gains processes are Q-martingales
\end{enumerate}
We make a discounted gains process by choosing a security price to use as
a numeraire. This follows from the intuition that all value is relative - we
used to use cows but now we use currency. A numeraire must be $> 0$ and must
not pay dividends.

\subsection{Spot Measure}

A natural choice in derivatives pricing is the bank account
$A_t=e^{\int r_udu}$. Discounted processes are denoted with a $*$. In the
Black-Scholes world this gives us
$$A_t^*=\frac{A_t}{A_t}=1$$
$$S_t^*=\frac{S_t}{A_t}$$
but $S_t^*$ is not a gains process
$$D_t^*=\int\frac{\delta_S(u,S_u)}{A_t}du$$
We must discount each dividend independently because they arrive at different
times and so require different discount rates.
$$G_t^*=S_t^*+D_t^*$$
The subscript on stochastic variables shows when the random variable becomes
observed and therefore a constant. $E_t[X]=E[X|\mathbb{F}_t]$

Well who cares? Let's assume we know $Q$ (which is unrealistic). Given equation
\ref{EMM} which says that all discounted gains process must be Q-martingales,
$c_t^*$ must be a Q-martingale.
$$c_t^*=E_t^Q[c_T^*]$$
but we want to know $c_t^*$
$$\frac{c_t}{A_t}=E_t^Q\left[ \frac{c_T}{A_T} \right] $$
$$c_t=E_t^Q\left[ \frac{A_t}{A_T}c_T^* \right] $$
We can throw $A_t$ inside the expectation operator because even if interest
rates are stochastic we can observe the interest rate at time $t$ so $A_t$ is
a constant.
$$c_t=E_t^Q\left[ e^{-\int_t^Tr_udu}max(S_T-K,0) \right] $$
now we can use Feynman-Kac. $Q$ allows us to discount at the riskless rate.

\subsubsection{Example: ZCB}

Payoff function
$$g(t,W_t)=\begin{cases} 1 & \text{if }t=T \\ 0 & \text{if }t\neq T\end{cases}$$
$P_{t,T}=PV_t(g(T))$ is the ex-dividend price. That is to say $P_{T,T}=0$
because we received the dividend.
$$P_{t,T}^*+D_t^*=E_T^Q[P_T^*+D_T^*]$$
but we want $P_{t,T}$
$$\frac{P_{t,T}}{A_t}+0=E_t^Q\left[ 0+\frac{1}{A_T} \right] $$
$D_T^*=\frac{1}{A_T}$ is not from $D_T^*=\frac{D_T}{A_T}$. It is from Dirac's
delta.
$$D_t=\begin{cases} 0 & \text{for }t<T \\ 1 & \text{for }t\geq T \end{cases}$$
$$D_T^*=\int_0^t\frac{1\delta_T(u)}{A_T}du$$
$$\delta_T(u)=\begin{cases}\infty & u=T \\ 0 & u\neq T\end{cases}$$
$$\int_0^{\infty}\delta_T(u)du=1$$
\textbf{Not really sure what has gone on here.}
But now we can move $A_t$ inside the expectation operator because it is
observed and therefore a constant
$$P_{t,T}=E_t^Q\left[ \frac{A_t}{A_T} \right] $$
We could also have defined the date $T$ payoff as the price at date $T$. Think
of it as selling the bond immediately before receiving the dividend.
$P_{T,T}=1$ and $D_t=0$
$$P_{t,T}^*+D_t^*=E_t^Q[P_{T,T}^*+D_T^*]$$
$$\frac{P_{t,T}}{A_t+0}=E_t^Q\left[ \frac{1}{A_T}+0\right] $$
$$P_{t,T}=E_t^Q\left[ \frac{A_t}{A_T}\right]$$
where
$$\frac{A_t}{A_T}=\frac{e^{\int_0^tr_udu}}{e^{\int_0^Tr_udu}}=
e^{\int_0^tr_udu-\int_0^Tr_udu}$$
$$P_{t,T}=E_t^Q\left[ e^{-\int_t^Tr_udu}\right]$$

\subsubsection{Example: Continuous Dividend}

A stock that continuously pays dividends where
$$g(t,S_t)=\begin{cases}\delta_S(t,S_t) & 0\leq t\leq T \\ 0 t>T\end{cases}$$
$P_T=0$ because dividends are paid out continuously so there is nothing to
sell.
$$D_t=\int_t^T\delta_S(u,S_u)du$$
To avoid arbitrage the discounted gains process must be be a Q-martingale
$$P_t^*+D_t^*=E_t^Q[P_T^*+D_T^*]$$
$$P_t^*=E_t^Q[D_T^*]$$
The discounted price is the expected value of discounted dividends received
between $t$ and $T$.
$$\frac{P_t}{A_t}=E_t^Q[D_T^*-D_t^*]$$
$$P_t=E_t^Q\left[ A_t\left( \int_t^T\frac{\delta_S(u,S_u)}{A_u}du
-\int_t^T\frac{\delta_S(u,S_u)}{A_u}du\right) \right]$$
$$P_t=E_t^Q\left[A_t\int_t^T\frac{\delta_S(u,S_u)}{A_u}du\right]$$
We can move the $A_t$ inside the integral because $A_t$ is observed and
therefore constant even if interest rates are stochastic.
$\frac{A_t}{A_T}=e^{-\int_t^Tr_udu}$
$$P_t=E_t^Q\left[ \int_t^Te^{-\int_t^Tr_udu}\delta_S(u,S_u)du\right]$$
Using the familiar dividend policy, $\delta_S(u,S_u)=\delta_S(u)S_u$
$$S_t=E_t^Q\left[ \int_t^Te^{-\int_t^Tr_udu}\delta_S(u)S_udu\right]$$
The discounted price is the expected value of discounted dividends received
between $t$ and $T$.

\subsection{Girsanov's Theorem}

Changing probabilities from $P$ to $Q$ means $W$ is no longer a standard Weiner
process. We can adjust the Weiner process
\begin{equation} \label{Girsanov}
d\tilde{W}_t=dW_t+\lambda_tdt
\end{equation}
where $\lambda_tdt$ is a deterministic increment. Under the correct $Q$,
$d\tilde{W}$ can be a standard Weiner process.
The fundamental theorem of asset pricing says that all discounted gains
processes must be $Q$-martingales. So we choose $Q$ to make the gains process
a martingale. We do this by making the drift, $\mu_Y$, equal to $0$. This boils
down to the following steps;
\begin{enumerate}
  \item Find dynamics of $G^*$ using Ito under probability measure $P$
  \item Change probability measure from $P$ to $Q$ with equation \ref{Girsanov}
  \item Choose $\lambda$ to make drift, $\mu$, equal to $0$
\end{enumerate}

\subsubsection{Example: Spot Measure Black-Scholes}

Using our spot measure where the numeraire asset is the bank account gave us;
$$dA_t=r_tA_tdt$$
$$dS_t=\mu_S(t)S_tdt+\sigma_S(t)S_tdW_t$$
$$dD_t=\delta_S(t)S_tdt$$
$$dG_t=dS_t+dD_t$$
We must ensure $G^*$ is a $Q$-martingale to avoid arbitrage. Starting with the
dynamics of $dS_t^*$ under the $P$ probability measure. We do not need to do
this step for the dividend process because $D$ has no Weiner process.
Identification:
$$S_t^*=\frac{S_t}{A_t}=e^{-\int_t^Tr_udu}S_t=f(t,S)$$
$$dS_t^*=\left[ f_1+f_2+\frac{1}{2}f_{22}(\sigma_SS)^2 \right] dt+
\sigma_Sf_2dW_t$$
Calculation:
$$f_1=e^{-\int_t^Tr_udu}\frac{\partial}{\partial t}
\left( -\int_t^Tr_udu\right) S_t$$
$S_t$ does not depend on $t$ it just has a subscript
$$=S_tA_t^{-1}\frac{\partial}{\partial t}-r_tdt=-r_tA_t^{-1}S_t$$
$A_t^{-1}$ does not depend on $S$
$$f_2=e^{-\int_t^Tr_udu}1=A_t^{-1}$$
$S$ dies in $f_2$ so there is nothing left to differentiate in $f_{22}$
$$f_{22}=0$$
Substitution:
$$dS_t^*=\left[ -r_tA_t^{-1}S_t+A_t^{-1}\mu_S(t)S_t+\frac{1}{2}0(\sigma_SS)^2
\right] dt+A_t^{-1}\sigma_S(t)S_tdW_t$$
Simplification: Remembering $S_t^*=\frac{S_t}{A_t}$
$$dS_t^*=\left[ \mu_S(t)S_t^*-r_tS_t^*\right] dt+\sigma_S(t)S_t^*dW_t$$
$$dS_t^*=\left[ \mu_S(t)-r_t\right] S_t^*dt+\sigma_S(t)S_t^*dW_t$$
This is only the dynamics of $S_t^*$ under $P$ probabilities with a standard
Weiner process.
We must make sure the discounted gains process $G^*$ is a $Q$-martingale.
$$dG_t^*=d(S_t^*+D_t^*)=dS_t^*+dD_t^*$$
$$=dS_t^*+d\int_0^t\frac{\delta_S(u)S_u}{A_u}du$$
Noticing $\frac{S_u}{A_u}=S_u^*$
$$=[\mu_S(t)-r_t]S_t^*dt+\sigma_S(t)S_t^*dW_t+\delta_S(t)S_t^*dt$$
Changing from $P$ to $Q$ probability measures with
$d\tilde{W}_t=dW_t+\lambda_tdt$
$$=[\mu_S(t)+\delta_S(t)-r_t]S_t^*dt+
\sigma_S(t)S_t^*(d\tilde{W}_t-\lambda(t)dt)$$
$$=[\mu_S(t)+\delta_S(t)-r_t-\sigma_S(t)\lambda(t)]S_t^*dt+
\sigma_S(t)S_t^*d\tilde{W}_t$$
In order to truly treat $d\tilde{W}_t$ as a standard Wiener and say we have
changed probabilities we need to choose $\lambda$ such that the drift of the
gains process $\mu_{G^*}=0$. Otherwise there would be arbitrage and much
wailing and gnashing of teeth.
$$\mu_S(t)+\delta_S(t)-r_t-\sigma_S(t)\lambda(t)=0$$
$$\lambda(t)=\frac{\mu_s(t)+\delta_S(t)-r_t}{\sigma_S(t)}$$
Where $\mu_S+\delta_S$ is the total return of the stock,
$\mu_S+\delta_S-r_t$ is the excess return and dividing the excess return by
$\sigma_S$ gives the Sharpe ratio.

We have identified $Q$ through its effect on $d\tilde{W}_t=dW_t+\lambda(t)$.
The payoff for an option, for example a call, is a function of $S_T$ and $K$,
not $S_t^*$ or any other process under $Q$ probabilities. We must determine
$S_T$ under $Q$ because
\begin{equation} \label{callExpVal}
c_t=E_t^Q\left[ e^{-\int_t^Tr_udu}max(S_T-K) \right]
\end{equation}
We found the dynamics of our gains process under $Q$ in order to find $\lambda$
so that we can plug this into our stock price process once we change to $Q$
probabilities. We used the gains process to find $\lambda$ because all gains
processes must be $Q$-martingales to avoid arbitrage. Having found $\lambda$ we
can use it to adjust processes to make $d\tilde{W}_t$ a standard Wiener.
$$dS_t=\mu_S(t)S_tdt+\sigma_S(t)S_t[d\tilde{W}_t-\lambda(t)dt]$$
$$=[\mu_S(t)+\sigma_S(t)\lambda(t)]S_tdt+\sigma_S(t)S_td\tilde{W}_t$$
substituting for $\lambda$
$$\left[\mu_S(t)+\sigma_S(t)\frac{\mu_S(t)+\delta_S(t)-r_t}{\sigma_S(t)}\right]
S_tdt+\sigma_S(t)d\tilde{W}_t$$
$$[r_t-\delta_S(t)]S_tdt+\sigma_S(t)S_td\tilde{W}_t$$
\textbf{No $\mu_S$ here} and we recognize it as a geometric Wiener process.
$$S_T=S_te^{\int_t^Tr(u)du-\delta_S(u)-\frac{1}{2}\sigma_S^2(u)du+
\int_t^T\sigma_S(u)d\tilde{W}_u}$$
and remembering \ref{callExpVal} we can use Feynman-Kac. We just change the
diffusion when using $Q$-probabilities. We have used the same replicating
portfolio logic to arrive here so we can use the same Black-Scholes-Merton
formula.

\subsection{Systemic Risk in Option Pricing}

Finding $\lambda$ is independent of derivative payoff we are valuing. The
underlying asset has systemic risk embodied by $\mu_S$ and $\delta_S$. They do
not show up in the valuation directly. It is not immediately intuitive why we
can price different derivatives, with different exposures to systemic risk,
using the same $\lambda$. Do all derivatives on $S$ have the same systemic
risk? \textbf{No}.

To understand how we can use $\lambda$ indifferently we need to look at the
relationship between $P$ and $Q$ measures.
$f_P$ and $f_Q$ are the probability densities associated with $P$ and $Q$,
respectively. We can say that $f_Q=zf_P$. Where
$$z=\frac{\partial Q}{\partial P}=
e^{-\frac{1}{2}\int_0^t\lambda_u^2du-\int_0^t\lambda_udW_u}$$
$z$ is a function of $\lambda$ and therefore of the Sharpe ratio under each
probability measure.
$$c_t=E_t^Q[c_T]$$
$$=\int_{-\infty}^{\infty}c_T(\varepsilon)f_Q(\varepsilon)d\varepsilon$$
$$=\int_{-\infty}^{\infty}c_T(\varepsilon)z(\varepsilon)
f_P(\varepsilon)d\varepsilon$$
$$=E_t^P[c_Tz]$$
But a product inside an expectation means there is covariance lurking.
$$=E_t^P[c_T]E_t^P[z]+Cov_t^P(c_T,z)$$
where $E_t^P[z]=1$
$$=E_t^P[c_T]+Cov_t^P(c_T,z)$$
The covariance is punishment and project specific. The $z$ may be the same
between projects but the cashflows can differ. No arbitrage accounts for the
$z$.

When we construct an EMM we are dragging the drift of a diffusion to $0$. Under
$P$ probabilities the diffusion is described by
$$S_t=S_0e^{(\mu-\frac{1}{2}\sigma^2)t-\sigma W_t}$$
and the expected value is
$$E_0^P[S_t]=S_0e^{\mu t}$$
When we change to $Q$ probabilities we drag the growth rate down
$$S_t=S_0e^{(r-\frac{1}{2}\sigma^2)t-\sigma\tilde{W}_t}$$
and the expected value drops to
$$E_0^Q[S_t]=S_0e^{rt}$$
We make the diffusion a martingale by dividing by the numeraire security
$$A_t=e^{rt}$$
$$E_0^Q\left[\frac{S_t}{A_t}\right]$$
$r$ is compensation for impatience while $\lambda\sigma$ is compensation for
risk.

\subsection{Alternative Numeraires}

So far we have discounted the price process by a riskless bank account.
$$A_t^*=\frac{A_t}{A_t}$$
$$G_t^*=S_t^*+D_t^*=\frac{S_t}{A_t}^*+\int_t^T\frac{\delta_S(u)}{A_u}du$$
We then chose $Q$ such that $G_t^*$ was a $Q$ martingale, $G_t^*=E_t^Q[G_T^*]$.
We don't have to use a bank account as the numeraire.
Depending on the goals, other securities may save some computation time and are
therefore better.

The requirements for a general numeraire security are; strictly positive and no
separate dividends, i.e. $B$ is a gains process for $B$.
$$A_t^B=\frac{A_t}{B_t}$$
$$G_t^B=S_t^B+D_t^B=\frac{S_t}{B_t}+\int_t^T\frac{\delta_S(u)}{B_u}du$$
We then chose $Q_B$ such that $G_t^B$ was a $Q$ martingale,
$G_t^B=E_t^{Q_B}[G_T^B]$.

\subsection{Forward Measure, $Q_T$}

$P(t,T)$ is the date $t$ price of a ZCB paying $1$ at time $T$. We define the
$T$ forward measure, $Q_T$, by requiring
$G_t^{P(t,T)}=E_t^{Q_T}[G_s^{P(t,T)}]$ for all $t\leq s$ for discounted gains
processes. That is to say, $Q_T$ is the probability measure that makes all
gains processes martingales.

\subsubsection{Example: EU Call}

$$c_t^{P(t,T)}=E_t^{Q_T}[c_T^{P(t,T)}]$$
iff
$$\frac{c_t}{P(t,T)}=E_t^{Q_T}\left[\frac{max(S_T-K,0)}{P(T,T)}\right]$$
$P(T,T)$ is $1$ by construction and $P(t,T)$ can be observed
$$c_t=E_t^{Q_T}[max(S_T-K,0)]P(t,T)$$
Who cares? This is the discounted expected value as before. But the forward
measure helps if interest rates are stochastic. Under the spot measure we would
have the expectation of a random variable times a random variable. This is
a covariance nightmare.

\subsection{Stock Measure, $Q_S$}

While the stock price process from our model is strictly positive, it does pay
dividends. We must adjust it slightly to use it as a numeraire. We use an
artificial security $S^c$ where all dividends are continuously reinvested. The
intuition is similar to the bank account. We accumulate units a the rate of
dividend payout (like we accumulate dollars at the risk free rate).
\begin{equation} \label{stockNumeraire}
  S_te^{\int_0^t\delta_S(u)du}
\end{equation}
Who cares? In short, $G^S$ will be $Q$-martingales by construction. More
thoroughly, We know our stock price process $S$ has a gains process of
$G=S+D$ and we know $S_t=S_0e^{\left(\mu-\frac{1}{2}\sigma^2\right) t+
\sigma W_t}$, $dS_t=\mu S_tdt+\sigma S_tdW_t$ and $dD_t=\delta S_tdt$.
Dividing through by $S_t$ removes the $\mu$. But the price process is still
growing at the dividend rate, $\delta$. Dividing by the adjusted stock price
process, $S^c$, removes $\mu$ and $\delta$ making $G^S$ a $Q$-martingale by
construction. 
While we are using the stock measure we can still use the bank account to find
the correct risk adjustment, $\lambda^S$.
$$A_t^S=\frac{A_t}{S_t^c}=\frac{1}{S_t}
\frac{e^{\int_0^tr_udu}}{e^{\int_0^t\delta_S(u)du}}$$
$$=\frac{1}{S_t}e^{\int_0^tr_u-\delta_S(u)du}$$
Now we need to find the dynamics of this diffusion, It\^o to the rescue.
Identification:
$$f(t,S)=\frac{1}{S_t}e^{\int_0^tr_u-\delta_S(u)du}$$
The $S$ means we need It\^o because this is a diffusion. Calculation:
$$f_1=\frac{1}{S_t}e^{\int_0^tr_u-\delta_S(u)du}(r_t-\delta_S(t))$$
$$f_2=-\frac{1}{S_t^2}e^{\int_0^tr_u-\delta_S(u)du}$$
$$f_{22}=2\frac{1}{S_t^3}e^{\int_0^tr_u-\delta_S(u)du}$$
Substitution (into \ref{IL}):
$$dA_t^S=$$
$$\left[ \frac{1}{S_t}e^{\int_0^tr_u-\delta_S(u)du}(r_t-\delta_S(t))-
\frac{1}{S_t^2}e^{\int_0^tr_u-\delta_S(u)du}\mu_S(t)S_t+
\frac{1}{2}2\frac{1}{S_t^3}e^{\int_0^tr_u-\delta_S(u)du}(\sigma_S(t)S_t)^2
\right] dt+$$
$$-\frac{1}{S_t^2}e^{\int_0^tr_u-\delta_S(u)du}\sigma_S(t)S_tdW_t$$
Simplification:
$$dA_t^S=\left[ \frac{A_t}{S_t^c}(r_t-\delta_S(t))-\frac{A_t}{S_t^c}\mu_S(t)+
\frac{A_t}{S_t^c}\sigma_S^2(t) \right] dt-\frac{A_t}{S_t^c}\sigma_S(t)dW_t$$
Adjust the Wiener process:
$$dA_t^S=A_t^S\left[ r_t-\delta_S(t)-\mu_S(t)+\sigma_S^2(t)\right] dt-
A_t^S\sigma_S(t)\left( dW_t^S-\lambda_t^Sdt\right)$$
$$dA_t^S=A_t^S\left[ r_t-\delta_S(t)-\mu_S(t)+\sigma_S^2(t)+
\sigma_S(t)\lambda_t^S\right] dt-
A_t^S\sigma_S(t)dW_t^S$$
is a $Q_S$-martingale iff no drift, i.e. set $dt$ term to $0$ and solve for
$\lambda$.
$$\lambda_t^S=\frac{-r_t+\delta_S(t)+\mu_S(t)-\sigma_S^2(t)}{\sigma_S(t)}$$
Now that we have nailed down $\lambda$ in the gains process we can use it to
adjust the stock price process from $P$ to $Q_S$ probabilities.
$$dS_t=\mu_S(t)S_tdt+\sigma_S(t)S_tdW_t$$
$$dS_t=\mu_S(t)S_tdt+\sigma_S(t)S_t(dW_t^S-\lambda_t^Sdt)$$
$$dS_t=\mu_S(t)S_tdt+\sigma_S(t)S_t\left[ dW_t^S-
\frac{-r_t+\delta_S(t)+\mu_S(t)-\sigma_S^2(t)}{\sigma_S(t)}\right]$$
$$dS_t(r_t-\delta_S(t)+\sigma_S^2(t))S_tdt+\sigma_S(t)S_tdW_t^S$$
We recognise this as a geometric Wiener.

\subsection{Example: Forward and Spot Measures Black-Scholes}

We can think of the payoff from an EU call option as paying the strike to
receive the stock at time $T$. We will pay the strike if the stock is worth
more.
$$c_T=max(S_T-K,0)=S_T1_{S_T>K}-K1_{S_T>K}$$
We will refer to the first term as $c_{1T}$ and the second as $c_{2T}$.
We can use the stock measure to find the present value of $c_{1T}$.
$$c_{1t}^S=E_t^{Q_S}[c_{1T}^S]$$
iff
$$\frac{c_{1t}}{S_t^c}=E_t^{Q_S}\left[\frac{S_T1_{S_T>K}}{S_T^c}\right]$$
$$\frac{c_{1t}}{S_te^{\int_0^t\delta_S(u)du}}=
E_t^{Q_S}\left[\frac{S_T}{S_Te^{\int_0^T\delta_S(u)du}}1_{S_T>K}\right]$$
$$c_{1t}=S_te^{\int_0^t\delta_S(u)du}\frac{1}{e^{\int_0^T\delta_S(u)du}}
E_t^{Q_S}\left[ 1_{S_T>K}\right]$$
$$c_{1t}=S_te^{-\int_t^T\delta_S(u)du}Q_S(S_T>K)$$
The expectation of an indicator function is the probability, under $Q_S$
probabilities here, of that event happening.
Remembering our solution for a geometric Wiener
$$S_T=S_te^{\int_t^Tr_u-\delta_S(u)+\sigma_S^2(u)-\frac{1}{2}\sigma_S^2(u)du+
\int_t^T\sigma_S(u)dW_u^S}=S_te^{m+\sigma\varepsilon}$$
Where $\varepsilon\sim N(0,1)$.
Note the adjusted Wiener process.
$$Q_S(S_T>K)=Pr(S_te^{m+\sigma\varepsilon}>K)$$
Solve for $\varepsilon$ inside $Pr$
$$Q_S(S_T>K)=
Pr\left(\varepsilon>\frac{ln\left(\frac{K}{S_t}\right) -m}{\sigma}\right)$$
Remembering section \ref{sec:BS} where we derived the Black-Scholes formula the
first time, we want to find the probability that $\varepsilon$ is less than
some value so we can use the cumulative normal density function. We flip the
inequality
$$Pr\left(\varepsilon>\frac{ln\left(\frac{K}{S_t}\right) -m}{\sigma}\right)=
Pr\left(\varepsilon<-\frac{ln\left(\frac{K}{S_t}\right) -m}{\sigma}\right)$$
$$=Pr\left(\varepsilon<\frac{ln\left(\frac{S_t}{K}\right) +m}{\sigma}\right)$$
$$=Pr\left(\varepsilon<\frac{ln\left(\frac{S_t}{K}\right) +
  \int_t^Tr_u-\delta_S(u)+\frac{1}{2}\sigma_S^2(u)du}
{\sqrt{\int_t^T\sigma_S^2(u)du}}\right)$$
Where $\frac{1}{2}\sigma_S^2=\sigma_S^2-\frac{1}{2}\sigma_S^2$. We
recognise this as $d_1$.
\begin{equation} \label{c1t}
  c_{1t}=S_te^{-\int_t^T\delta_S(u)du}N(d_1)
\end{equation}

We can use the forward measure to find the present value of $c_{2T}$.
$$c_{2t}^{P(t,T)}=E_t^{Q_T}[c_{2T}]$$
$$\frac{c_{2t}}{P(t,T)}=KE_t^{Q_T}[1_{S_T}>K]$$
$$c_{2t}=P(t,T)Q_T(S_T>K)$$
In the Black-Scholes world, interest rates are deterministic so $Q_T$ = $Q$.
$$Q_T(S_T>K)=Q(S_T>K)=N(d_2)$$
Remembering $P(t,T)=e^{-\int_t^Tr_udu}$
\begin{equation} \label{c2t}
  c_{2t}=Ke^{-\int_t^Tr_udu}N(d_2)
\end{equation}

\end{document}

