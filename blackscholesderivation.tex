
\section{Black-Scholes Derivation} \label{sec:BS}

Recognizing the dynamics (equation \ref{feynkacParDyn}) is a geometric Wiener
process we can write the solution to the PDE as
\begin{equation} \label{BSPDE}
    Z_T=Z_te^{\int_t^T\left[ r(u)-\delta(u) \right]
    -\frac{1}{2}\sigma_S^2(u)du
    + \int_t^T\sigma_S(u)dW_u}
\end{equation}
This is messy but we can simplify notation.
\[\mu=\int_t^Tr(u)-\delta_S(u)-\frac{1}{2}\sigma_S^2(u)du\]
\[\sigma=\sqrt{\int_t^T\sigma_S^2(u)du}\]
\[Z_T=Z_te^{\mu+\sigma\varepsilon}\]
where
\[\varepsilon \sim N(0,1)\]
This is a simpler way of writing $Z_T$ but means the same thing. It is the risk
adjusted stock price process discounted at the risk-free rate.
We are still left trying to solve equation \ref{feynkacPar}.
Remembering $r$ is deterministic, we can take it outside the expectation.
\[e^{-\int_t^Tr(u)du}E[\text{max}(Z_T-K,0)]\]
The remaining $\text{max}(Z_T-K,0)$ has a kink.
This matters because equations with kinks are non-differentiable.
Writing with indicator functions solves this problem.
We can take the expected value of the indicator functions separately.
\[E\left[ (Z_T-K)1_{Z_T\geq K} \right]\]
\[E\left[ Z_T1_{Z_T\geq K} - K1_{Z_T\geq K} \right]\]
\begin{equation} \label{BSind}
    E\left[ Z_T1_{Z_T\geq K} \right] - KE\left[ 1_{Z_T\geq K} \right]
\end{equation}
The expected value of an indicator function is the probability the event
will happen.
Tackling the second part of equation \ref{BSind} first:
\[
    E\left[ 1_{Z_T\geq K}\right] =Pr(Z_T\geq K) =
    Pr(Z_te^{\mu +\sigma\varepsilon}\geq K)
\]
Working with normals is much nicer than working with lognormals. We can solve
for $\varepsilon$ in the inequality. It is also important to note
that \textbf{by substituting $Z_T$ for $Z_te^{\mu+\sigma\varepsilon}$ we
observe $Z_t$ at time $t$.}
This means is it a constant rather than a stochastic variable.
\[Z_te^{\mu +\sigma\varepsilon}\geq K\]
\[e^{\mu +\sigma\varepsilon}\geq\frac{K}{Z_t}\]
\[\mu +\sigma\varepsilon\geq ln\left( \frac{K}{Z_t} \right)\]
\[\varepsilon\geq\frac{ln\left( \frac{K}{Z_t} \right)-\mu}{\sigma}\]
To use the cumulative normal density function we want to get the probability
that $\varepsilon$ is below a value not above.
This follows from:
\[
    E\left[ 1_{\varepsilon\leq K} \right] =
    \int_{-\infty}^{\infty} 1_{\varepsilon\leq K} n(\varepsilon)d\varepsilon
\]
where $n(\varepsilon)$ is the normal probability density function.
We end up here because if $X$ is continuous,
it is not but we make it so with an indicator function,
then the expectation of $g(X)$ is defined as
\[E[g(X)]=\int_{-\infty}^{\infty}g(x)f(x)dx\]
where $f$ is the probability density function of $X$.
We can now flip the inequality and put a minus sign on the RHS.
\[
    Pr\left( \varepsilon\leq -\frac{ln\left(\frac{K}{Z_t}\right)
    -\mu}{\sigma}\right)
\]
Minus sign means we flip the $K$ and $Z_T$ because
\[
    -ln\left(\frac{a}{b}\right) =-(ln(a)-ln(b))=ln(b)-ln(a)=
    ln\left(\frac{a}{b}\right)
\]
\[
    Pr\left( \varepsilon\leq\frac{ln\left(\frac{Z_t}{K}\right)
    +\mu}{\sigma}\right)
\]
\[
    E\left[ 1_{\varepsilon\leq K} \right] =
    N\left( \frac{ln\left(\frac{Z_t}{K}\right)
    +\mu}{\sigma}\right) =N(-L)
\]
Now tackling the second part of equation \ref{BSind}:
\[
    E\left[ Z_T1_{Z_T\geq K} \right] =
    \int_{-\infty}^{\infty}Z_te^{\mu+\sigma\varepsilon}1_{Z_T\geq
    K}n(\varepsilon)d\varepsilon
\]
We have found
$\int_{-\infty}^{\infty} 1_{\varepsilon\leq K}n(\varepsilon)d\varepsilon=
\frac{ln\left(\frac{K}{Z_t}\right)-\mu}{\sigma}$.
We called this expression $L$ and now we set it as the lower limit of the
integral \textbf{why?}, similar to setting $K$ as the upper limit when finding
the expectation of the raw indicator function --
this time we set $L$ as the lower limit because the inequality sign of the
indicator functions are opposite.
\[\int_{L}^{\infty}Z_te^{\mu+\sigma\varepsilon}n(\varepsilon)d\varepsilon\]
Remembering that $Z_t$ is observed at time $t$ and that $n(e)$ is the normal
density function.
$n(e)=\frac{1}{\sqrt{2\pi}}e^{-\frac{\varepsilon^2}{2}}$.
\[
    Z_t\int_{L}^{\infty}e^{\mu+\sigma\varepsilon}
    \frac{1}{\sqrt{2\pi}}e^{-\frac{\varepsilon^2}{2}}d\varepsilon
\]
We can combine $\varepsilon$ by completing the square
$(a+b)^2=a^2+2ab+b^2$.
\[
    \frac{1}{2}(2\mu+2\sigma\varepsilon-\frac{1}{2}\varepsilon^2)=
    \frac{1}{2}(2\mu-(\sigma-\varepsilon)^2+\sigma^2)
\]
Therefore
\[
    Z_t\int_{L}^{\infty}e^{\mu+\frac{1}{2}\sigma^2}
    \frac{1}{\sqrt{2\pi}}e^{-\frac{1}{2}(\varepsilon-\sigma)^2}d\varepsilon
\]
We still need the standard normal density in order to use our $N(\cdot)$.
We can set $u=\varepsilon-\sigma$ therefore, $du=d\varepsilon$ because $\sigma$
is constant.
When $\varepsilon=L$, $u=L-\sigma$.
When $\varepsilon=\infty$, $u=\infty-\sigma$.
Now
\[
    Z_te^{\mu+\frac{1}{2}\sigma^2}
    \int_{L-\sigma}^{\infty}\frac{1}{\sqrt{2\pi}}e^{-\frac{1}{2}u^2}du
\]
\[Z_te^{\mu+\frac{1}{2}\sigma^2}\int_{L-\sigma}^{\infty}n(u)du\]
Now we have our standard normal $\int_{L-\sigma}^{\infty}n(u)du$ and we can
switch the inequality direction as we did above.
\[Pr(>L-\sigma)=Pr(<-(L-\sigma))\]
\[Z_te^{\mu+\frac{1}{2}\sigma^2}\int_{\infty}^{-(L-\sigma)}n(u)du\]
\[
    E\left[ Z_T1_{Z_T\geq K}\right] =
    Z_te^{\mu+\frac{1}{2}\sigma^2}N(-(L-\sigma))
\]
Putting this all back together
\begin{equation}
    c(t,S)=e^{-\int_t^Tr(u)du}
    \left[ Z_te^{\mu+\frac{1}{2}\sigma^2}N(-(L-\sigma)) -KN(-L) \right]
\end{equation}
where; $Z_t=S$,
\[L=\frac{ln\left(\frac{K}{Z_t}\right)-\mu}{\sigma}\]
\[\mu=\int_t^Tr(u)-\delta_S(u)-\frac{1}{2}\sigma_S^2(u)du\]
\[\sigma=\sqrt{\int_t^T\sigma_S^2(u)du}\]
We have been modeling a $Z$ process and not $S$. Option pricing is computing
present values but by adjusting the expected payoff rather than the discount
rate. While $S$ grows at $\mu$, $Z$ grows at $r$.
We can substitute to get to the more familiar Black-Scholes model.
Remembering our simplifying notation for $\mu$ and $\sigma$
\[\mu=\int_t^Tr(u)-\delta_S(u)-\frac{1}{2}\sigma_S^2(u)du\]
\[\sigma=\sqrt{\int_t^T\sigma_S^2(u)du}\]
\[
    c(t,S)=
    e^{-\int_t^Tr(u)du+\int_t^Tr(u)du-\delta_S(u)-\frac{1}{2}\sigma_S^2(u)du
    +\int_t^T\frac{1}{2}\sigma_S^2(u)du}
    Z_tN(-(L-\sigma))-e^{\int_t^Tr(u)du}KN(-L)
\]
\begin{equation} \label{BS}
    e^{-\delta_S(u)du}Z_tN(-(L-\sigma))-e^{\int_t^Tr(u)du}KN(-L)
\end{equation}
Only $d_1$ and $d_2$ remain unfamiliar now.
\[
    -(L-\sigma)=-L+\sigma=-\frac{ln\left(\frac{K}{Z_t}\right)-\mu}{\sigma}+
    \sqrt{\int_t^T\sigma_S^2(u)du}
\]
\[
    \frac{ln\left(\frac{Z_t}{K}\right)+\int_t^Tr(u)-\delta_S(u)-
    \frac{1}{2}\sigma_S^2(u)du}{\sqrt{\int_t^T\sigma_S^2(u)du}}+
    \frac{\int_t^T\sigma_S^2(u)du}{\sqrt{\int_t^T\sigma_S^2(u)du}}
\]
\begin{equation} \label{BSd1}
    d_1=\frac{ln\left(\frac{Z_t}{K}\right)+\int_t^Tr(u)-\delta_S(u)+
    \frac{1}{2}\sigma_S^2(u)du}{\sqrt{\int_t^T\sigma_S^2(u)du}}
\end{equation}
\[-L=-\frac{ln\left(\frac{K}{Z_t}\right)-\mu}{\sigma}\]
\begin{equation} \label{BSd2}
    d_2=\frac{ln\left(\frac{Z_t}{K}\right)+\int_t^Tr(u)-\delta_S(u)-
    \frac{1}{2}\sigma_S^2(u)du}{\sqrt{\int_t^T\sigma_S^2(u)du}}
\end{equation}
To get the Black-Scholes formula with constant parameters,
we need to remember the rules for integrating a constant;
$\int_a^bkdx=k\int_a^b1dx=k(b-a)$.
\[c(t,S)=e^{-\delta_S(T-t)}SN(d_1)-e^{-r(T-t)}KN(d_2)\]
\[
    d_1=\frac{ln\left(\frac{S}{K}\right) +
    (r-d_S+\frac{1}{2}\sigma_S^2)(T-t)}{\sigma_S\sqrt{T}}
\]
\[
    d_2=\frac{ln\left( \frac{S}{K} \right) +
    (r-d_S-\frac{1}{2}\sigma_S^2)(T-t)}{\sigma_S\sqrt{T}}
\]
The most important point here is the \textbf{replicating portfolio}.
