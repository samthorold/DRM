
\section{Dividends}

$\phi$ is the number of stocks. So $\phi_t S_t$ is the value of the stock
position at date t. The value of the position from time $0$ to $t$ =
$\int_0^t \phi_u dS_u$. We have a model for $dS_u$.

Unfortunately, we are forgetting about one common aspect of stocks, dividends.
Since we work in continuous time we will have to model dividends as a
constant drip of income, at a deterministic rate $\delta_S(t)$,  rather than
discrete lumps of cash. While this is not all that realistic for individual
stocks, the model does make sense for a security such as an index fund which
would pay out in something of a drip. There will be a stock in the index
paying some small amount of dividend compared to the value of the index. This
means $S_t$ is an ex-dividend price, we have not taken into account the gains
(or foregone gains) from dividends.
\[G_t=S_t+D_t\]
where
\[D_t=\int_0^t \delta_S(u)S_udu\]
and
\[dG_t=dS_t+dD_t\]
which is a mix of present and accounting values. Over small time interval
$dt$ we don't mind.
\[dG_t=\mu_S(t)S_tdt+\sigma_S(t)S_tdW_t+\delta_S(t)S_tdt\]
\[dG_t=\left[ \mu_S(t)+\delta_S(t) \right] S_tdt+\sigma_S(t)dW_t\]
$G_t$ does not assume optimal use of dividends, i.e. reinvestment. Since we
work with $dG_u$ this is fine.

\section{Buy and Hold Strategy}

Assume we invest in a bank account and a stock. Buy and hold so the quantities
of the bank account and the stock are fixed from time $0$ to $t$. We invest
$\theta$ quantity of \$1 bank accounts and $\phi$ quantity of $S$. The bank
account process is
\[A_t=e^{\int_0^t r(u)du}\]
\[dA_t=e^{\int_0^t r(u)du} \int_0^t r(u)du\]
\[dA_t=A_tr_td\]
So the portfolio value from $0$ to $t$ is
\[\int_0^t \phi_u dG_u + \int_0^t \theta_udA_u\]

\section{Self-financing Portfolio}

Remembering the binomial intuition, we need a self-financing portfolio to
replicate the option payoff.
\begin{equation} \label{replport}
    \phi_tS_t+\theta_tA_t=\phi_0S_0+\theta_0A_0+ \int_0^t \phi_u dG_u +
    \int_0^t \theta_udA_u
\end{equation}
We don't know how to solve integrals that don't have regular $dt$ or $du$
terms.
